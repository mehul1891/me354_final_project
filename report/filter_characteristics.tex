% !TEX root = ./proj_report.tex
\graphicspath{{mehul_pics/}}% Set graphics path location

\section{Filter Performance}
\noindent {\bf Image processing and degradation model: }\\
Blurred images are modeled as the true image convolved with a point spread function and additive noise. The point spread function (PSF) that convolves the true image is generally a property associated with the optics that have contributed to the blur while noise can be due to poor/excessive illumination, quantization errors and other sources. Mathematically, a blurry image can be represented as follows\\
\begin{equation}
v(m,n)= h(m,n) \star u(m,n) + \eta(m,n)
\end{equation}
where, $u(m,n)$ is the true image, $h(m,n)$ is the PSF, $\eta(m,n)$ is the additive noise and $v(m,n)$ is the blurry image. Image sharpening involves calculating an estimate of the true image $\hat{u}(m,n)$ using filters.
\begin{equation}
\hat{u}(m,n)= g(m,n) \star v(m,n)
\end{equation}
Filters calculate a function $g(m,n)$ using some knowledge of the PSF that caused the blur and an estimate of the signal to noise ratio.\\

\noindent {\bf Filter kernels: }\\
Choice of filter is very important in image restoration and retrieved image characteristics. This part of the project aims to study and compare different filters based on enhancement effectiveness and de-blur characteristics under various blur sources. The filters use for this study are: 
\begin{enumerate}
\item Inverse and Pseudo-inverse filter
\item Wiener filter
\item Geometric mean filter
\item Constrained least squares filter
\end{enumerate} 
Mathematical formulation of these filters has been presented in the Appendix and will not be discussed here. 
\subsection{Train Image: Supersonic flow around a sphere}
In this simulated motion blur, the train image was procured from Van Dyke's Album of Fluid Motion, and depicts the formation of shock around a sphere at flow Mach no. 3. This image was blurred using a "motion" PSF and had Gaussian white additive noise. The blurred and recovered images as well and a brief description of the filter characteristics are presented in Figure~\ref{fig:sphere}.

\begin{figure}
        \centering
        \begin{subfigure}[b]{0.4\textwidth}
                \centering
                \includegraphics[width=\textwidth]{ssphere_wiener.jpg}
                \caption{Wiener filter sharpened image}
               
        \end{subfigure}
        \begin{subfigure}[b]{0.4\textwidth}
                \centering
                \includegraphics[width=\textwidth]{ssphere_motion.jpg}
                \caption{Blurry image} 
        \end{subfigure} 
       
        \caption{Image enhancement on the blurred image was performed using all the listed filters. For representative purposes only the sharpened image using the Wiener filter has been shown here. The blur PSF used as the initial condition for the filtering process is "motion". Variance of the additive noise is of $O(10^{-5})$. } \label{fig:sphere}
\end{figure}

The Wiener filter was found to perform the best at reconstructing the true optical kernel, this conclusion was reached by comparing each of the filter approximated kernels with the kernel used to blur the image. The metric used for this comparison was the 2-norm of the difference between the 2D projections of the optical filter obtained kernel and the kernel used to blur the image. A plot of the kernels can be seen is shown in Figure~\ref{fig:sph_kernels}.   

\begin{figure}[h!]
  \centering
                \centering
                \includegraphics[width=.5\textwidth]{kernel_motion.jpg}
                \caption{Kernel comparison for blur due to motion}
 \label{fig:sph_kernels}
\end{figure}

Although this approach has proven to be very useful for the study of training images, the use of it with real blurred images is very challenging. Some of the draw backs encountered with this approach is its dependence on a-priory knowledge of the blur cause, as well as noise sources and strengths. 

\subsection{Real Image: Barn and Barrels}

To test this approach with real blurred images, two images of the same source have been captured, one on focus and one out of focus, the images are shown on Figure~\ref{fig:barrels}. With the use of a de-convolution, the true kernel was obtained. To reduce the effect of noise in the kernel, both images were split in multiple parts, each pair was used to obtain a kernel independently, and then averaged to obtain an approximation of the true kernel were noise has been reduced. A similar analysis as the one done for the sphere was performed and a comparison between the true and filter obtained kernels can be seen in Figure~\ref{fig:true_kernel}.

\begin{figure}
        \centering
        \begin{subfigure}[b]{0.4\textwidth}
                \centering
                \includegraphics[width=\textwidth]{DSC_0517.jpg}
                \caption{Sharp image}
                
        \end{subfigure}
        \begin{subfigure}[b]{0.4\textwidth}
                \centering
                \includegraphics[width=\textwidth]{DSC_0518.jpg}
                \caption{Blurry image} 
        \end{subfigure}
 \label{fig:barrels}
\end{figure}

\begin{figure}[h!]

  \centering
                \centering
                \includegraphics[width=.5\textwidth]{true_kernel.jpg}
                \caption{All previously mentioned filters have been used to attempt to approximate the true optical kernel. The blur PSF used as the initial condition for the filtering process is "disk". Variance of the additive noise is of $O(10^{-3.5})$ }
                \label{fig:true_kernel}
\end{figure}
\newpage  
As can be seen in Figure~\ref{fig:true_kernel}, the approximations for the kernel are not very good, and without knowing the true cause of the blur, it is very challenging and time consuming to find a filter that will be able to sharpen the image. Due to the presence of undetermined noise, we can't just use a simple norm to determine the effectiveness of a filter type anymore. To be able to deal with real blurred images sharpness metrics are required.